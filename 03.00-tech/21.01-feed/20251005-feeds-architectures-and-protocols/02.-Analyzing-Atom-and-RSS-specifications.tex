% Options for packages loaded elsewhere
\PassOptionsToPackage{unicode}{hyperref}
\PassOptionsToPackage{hyphens}{url}
\PassOptionsToPackage{dvipsnames,svgnames,x11names}{xcolor}
%
\documentclass[
  letterpaper,
  DIV=11,
  numbers=noendperiod]{scrartcl}

\usepackage{amsmath,amssymb}
\usepackage{iftex}
\ifPDFTeX
  \usepackage[T1]{fontenc}
  \usepackage[utf8]{inputenc}
  \usepackage{textcomp} % provide euro and other symbols
\else % if luatex or xetex
  \usepackage{unicode-math}
  \defaultfontfeatures{Scale=MatchLowercase}
  \defaultfontfeatures[\rmfamily]{Ligatures=TeX,Scale=1}
\fi
\usepackage{lmodern}
\ifPDFTeX\else  
    % xetex/luatex font selection
\fi
% Use upquote if available, for straight quotes in verbatim environments
\IfFileExists{upquote.sty}{\usepackage{upquote}}{}
\IfFileExists{microtype.sty}{% use microtype if available
  \usepackage[]{microtype}
  \UseMicrotypeSet[protrusion]{basicmath} % disable protrusion for tt fonts
}{}
\makeatletter
\@ifundefined{KOMAClassName}{% if non-KOMA class
  \IfFileExists{parskip.sty}{%
    \usepackage{parskip}
  }{% else
    \setlength{\parindent}{0pt}
    \setlength{\parskip}{6pt plus 2pt minus 1pt}}
}{% if KOMA class
  \KOMAoptions{parskip=half}}
\makeatother
\usepackage{xcolor}
\setlength{\emergencystretch}{3em} % prevent overfull lines
\setcounter{secnumdepth}{5}
% Make \paragraph and \subparagraph free-standing
\makeatletter
\ifx\paragraph\undefined\else
  \let\oldparagraph\paragraph
  \renewcommand{\paragraph}{
    \@ifstar
      \xxxParagraphStar
      \xxxParagraphNoStar
  }
  \newcommand{\xxxParagraphStar}[1]{\oldparagraph*{#1}\mbox{}}
  \newcommand{\xxxParagraphNoStar}[1]{\oldparagraph{#1}\mbox{}}
\fi
\ifx\subparagraph\undefined\else
  \let\oldsubparagraph\subparagraph
  \renewcommand{\subparagraph}{
    \@ifstar
      \xxxSubParagraphStar
      \xxxSubParagraphNoStar
  }
  \newcommand{\xxxSubParagraphStar}[1]{\oldsubparagraph*{#1}\mbox{}}
  \newcommand{\xxxSubParagraphNoStar}[1]{\oldsubparagraph{#1}\mbox{}}
\fi
\makeatother

\usepackage{color}
\usepackage{fancyvrb}
\newcommand{\VerbBar}{|}
\newcommand{\VERB}{\Verb[commandchars=\\\{\}]}
\DefineVerbatimEnvironment{Highlighting}{Verbatim}{commandchars=\\\{\}}
% Add ',fontsize=\small' for more characters per line
\usepackage{framed}
\definecolor{shadecolor}{RGB}{241,243,245}
\newenvironment{Shaded}{\begin{snugshade}}{\end{snugshade}}
\newcommand{\AlertTok}[1]{\textcolor[rgb]{0.68,0.00,0.00}{#1}}
\newcommand{\AnnotationTok}[1]{\textcolor[rgb]{0.37,0.37,0.37}{#1}}
\newcommand{\AttributeTok}[1]{\textcolor[rgb]{0.40,0.45,0.13}{#1}}
\newcommand{\BaseNTok}[1]{\textcolor[rgb]{0.68,0.00,0.00}{#1}}
\newcommand{\BuiltInTok}[1]{\textcolor[rgb]{0.00,0.23,0.31}{#1}}
\newcommand{\CharTok}[1]{\textcolor[rgb]{0.13,0.47,0.30}{#1}}
\newcommand{\CommentTok}[1]{\textcolor[rgb]{0.37,0.37,0.37}{#1}}
\newcommand{\CommentVarTok}[1]{\textcolor[rgb]{0.37,0.37,0.37}{\textit{#1}}}
\newcommand{\ConstantTok}[1]{\textcolor[rgb]{0.56,0.35,0.01}{#1}}
\newcommand{\ControlFlowTok}[1]{\textcolor[rgb]{0.00,0.23,0.31}{\textbf{#1}}}
\newcommand{\DataTypeTok}[1]{\textcolor[rgb]{0.68,0.00,0.00}{#1}}
\newcommand{\DecValTok}[1]{\textcolor[rgb]{0.68,0.00,0.00}{#1}}
\newcommand{\DocumentationTok}[1]{\textcolor[rgb]{0.37,0.37,0.37}{\textit{#1}}}
\newcommand{\ErrorTok}[1]{\textcolor[rgb]{0.68,0.00,0.00}{#1}}
\newcommand{\ExtensionTok}[1]{\textcolor[rgb]{0.00,0.23,0.31}{#1}}
\newcommand{\FloatTok}[1]{\textcolor[rgb]{0.68,0.00,0.00}{#1}}
\newcommand{\FunctionTok}[1]{\textcolor[rgb]{0.28,0.35,0.67}{#1}}
\newcommand{\ImportTok}[1]{\textcolor[rgb]{0.00,0.46,0.62}{#1}}
\newcommand{\InformationTok}[1]{\textcolor[rgb]{0.37,0.37,0.37}{#1}}
\newcommand{\KeywordTok}[1]{\textcolor[rgb]{0.00,0.23,0.31}{\textbf{#1}}}
\newcommand{\NormalTok}[1]{\textcolor[rgb]{0.00,0.23,0.31}{#1}}
\newcommand{\OperatorTok}[1]{\textcolor[rgb]{0.37,0.37,0.37}{#1}}
\newcommand{\OtherTok}[1]{\textcolor[rgb]{0.00,0.23,0.31}{#1}}
\newcommand{\PreprocessorTok}[1]{\textcolor[rgb]{0.68,0.00,0.00}{#1}}
\newcommand{\RegionMarkerTok}[1]{\textcolor[rgb]{0.00,0.23,0.31}{#1}}
\newcommand{\SpecialCharTok}[1]{\textcolor[rgb]{0.37,0.37,0.37}{#1}}
\newcommand{\SpecialStringTok}[1]{\textcolor[rgb]{0.13,0.47,0.30}{#1}}
\newcommand{\StringTok}[1]{\textcolor[rgb]{0.13,0.47,0.30}{#1}}
\newcommand{\VariableTok}[1]{\textcolor[rgb]{0.07,0.07,0.07}{#1}}
\newcommand{\VerbatimStringTok}[1]{\textcolor[rgb]{0.13,0.47,0.30}{#1}}
\newcommand{\WarningTok}[1]{\textcolor[rgb]{0.37,0.37,0.37}{\textit{#1}}}

\providecommand{\tightlist}{%
  \setlength{\itemsep}{0pt}\setlength{\parskip}{0pt}}\usepackage{longtable,booktabs,array}
\usepackage{calc} % for calculating minipage widths
% Correct order of tables after \paragraph or \subparagraph
\usepackage{etoolbox}
\makeatletter
\patchcmd\longtable{\par}{\if@noskipsec\mbox{}\fi\par}{}{}
\makeatother
% Allow footnotes in longtable head/foot
\IfFileExists{footnotehyper.sty}{\usepackage{footnotehyper}}{\usepackage{footnote}}
\makesavenoteenv{longtable}
\usepackage{graphicx}
\makeatletter
\newsavebox\pandoc@box
\newcommand*\pandocbounded[1]{% scales image to fit in text height/width
  \sbox\pandoc@box{#1}%
  \Gscale@div\@tempa{\textheight}{\dimexpr\ht\pandoc@box+\dp\pandoc@box\relax}%
  \Gscale@div\@tempb{\linewidth}{\wd\pandoc@box}%
  \ifdim\@tempb\p@<\@tempa\p@\let\@tempa\@tempb\fi% select the smaller of both
  \ifdim\@tempa\p@<\p@\scalebox{\@tempa}{\usebox\pandoc@box}%
  \else\usebox{\pandoc@box}%
  \fi%
}
% Set default figure placement to htbp
\def\fps@figure{htbp}
\makeatother

\KOMAoption{captions}{tableheading}
\makeatletter
\@ifpackageloaded{caption}{}{\usepackage{caption}}
\AtBeginDocument{%
\ifdefined\contentsname
  \renewcommand*\contentsname{Table of contents}
\else
  \newcommand\contentsname{Table of contents}
\fi
\ifdefined\listfigurename
  \renewcommand*\listfigurename{List of Figures}
\else
  \newcommand\listfigurename{List of Figures}
\fi
\ifdefined\listtablename
  \renewcommand*\listtablename{List of Tables}
\else
  \newcommand\listtablename{List of Tables}
\fi
\ifdefined\figurename
  \renewcommand*\figurename{Figure}
\else
  \newcommand\figurename{Figure}
\fi
\ifdefined\tablename
  \renewcommand*\tablename{Table}
\else
  \newcommand\tablename{Table}
\fi
}
\@ifpackageloaded{float}{}{\usepackage{float}}
\floatstyle{ruled}
\@ifundefined{c@chapter}{\newfloat{codelisting}{h}{lop}}{\newfloat{codelisting}{h}{lop}[chapter]}
\floatname{codelisting}{Listing}
\newcommand*\listoflistings{\listof{codelisting}{List of Listings}}
\makeatother
\makeatletter
\makeatother
\makeatletter
\@ifpackageloaded{caption}{}{\usepackage{caption}}
\@ifpackageloaded{subcaption}{}{\usepackage{subcaption}}
\makeatother

\usepackage{bookmark}

\IfFileExists{xurl.sty}{\usepackage{xurl}}{} % add URL line breaks if available
\urlstyle{same} % disable monospaced font for URLs
\hypersetup{
  pdftitle={Analyzing Atom and RSS Specifications},
  pdfauthor={Dario Airoldi},
  pdfkeywords={RSS 2.0, Atom Syndication, Feed Specifications, WebSub
Protocol, Push Notifications, Pull Mechanisms, Feed Metadata},
  colorlinks=true,
  linkcolor={blue},
  filecolor={Maroon},
  citecolor={Blue},
  urlcolor={Blue},
  pdfcreator={LaTeX via pandoc}}


\title{Analyzing Atom and RSS Specifications}
\author{Dario Airoldi}
\date{2025-10-10}

\begin{document}
\maketitle

\renewcommand*\contentsname{Table of contents}
{
\hypersetup{linkcolor=}
\setcounter{tocdepth}{3}
\tableofcontents
}

\section{📊 Analyzing Atom and RSS
Specifications}\label{analyzing-atom-and-rss-specifications}

\begin{quote}
A deep dive into the data structures, notification mechanisms, and
architectural differences between the two dominant feed syndication
standards.
\end{quote}

\subsection{📋 Table of Contents}\label{table-of-contents}

\begin{enumerate}
\def\labelenumi{\arabic{enumi}.}
\tightlist
\item
  \hyperref[-introduction]{🎯 Introduction}
\item
  \hyperref[-rss-20-specification-analysis]{📰 RSS 2.0 Specification
  Analysis}
\item
  \hyperref[-atom-specification-analysis]{⚛️ Atom Specification
  Analysis}
\item
  \hyperref[-comparative-analysis]{⚖️ Comparative Analysis}
\item
  \hyperref[-references]{📚 References}
\end{enumerate}

\begin{center}\rule{0.5\linewidth}{0.5pt}\end{center}

\subsection{🎯 Introduction}\label{introduction}

Feed syndication has become a cornerstone of content distribution on the
web, with \textbf{RSS 2.0} and \textbf{Atom} representing the two
primary standards. While both serve similar purposes---enabling
efficient content distribution and updates---they differ significantly
in their data models, notification mechanisms, and philosophical
approaches to standardization.

This analysis examines:

\begin{itemize}
\tightlist
\item
  \textbf{Data structures} and available metadata fields
\item
  \textbf{Notification mechanisms} (push vs.~pull)
\item
  \textbf{Protocol support} and implementation patterns
\item
  \textbf{Key architectural differences} between the specifications
\end{itemize}

\begin{center}\rule{0.5\linewidth}{0.5pt}\end{center}

\subsection{📰 RSS 2.0 Specification
Analysis}\label{rss-2.0-specification-analysis}

\subsubsection{Overview}\label{overview}

\textbf{RSS 2.0} (Really Simple Syndication) is the most widely adopted
feed format, particularly in podcasting and blog syndication. Developed
by UserLand Software and published in 2002, RSS 2.0 emphasizes
simplicity and backward compatibility.

\begin{quote}
📖 \textbf{Specification}: RSS 2.0 is defined in the
\href{https://cyber.harvard.edu/rss/rss.html}{RSS 2.0 Specification}
maintained by Harvard's Berkman Center.
\end{quote}

\begin{center}\rule{0.5\linewidth}{0.5pt}\end{center}

\subsubsection{📦 Data Available from RSS
Notifications}\label{data-available-from-rss-notifications}

RSS 2.0 provides a hierarchical structure with channel-level and
item-level metadata.

\paragraph{\texorpdfstring{\textbf{Channel-Level Data} (Feed
Metadata)}{Channel-Level Data (Feed Metadata)}}\label{channel-level-data-feed-metadata}

Channel elements describe the overall feed and apply to all items within
it.

\begin{longtable}[]{@{}
  >{\raggedright\arraybackslash}p{(\linewidth - 8\tabcolsep) * \real{0.1556}}
  >{\raggedright\arraybackslash}p{(\linewidth - 8\tabcolsep) * \real{0.1333}}
  >{\raggedright\arraybackslash}p{(\linewidth - 8\tabcolsep) * \real{0.2222}}
  >{\raggedright\arraybackslash}p{(\linewidth - 8\tabcolsep) * \real{0.2889}}
  >{\raggedright\arraybackslash}p{(\linewidth - 8\tabcolsep) * \real{0.2000}}@{}}
\toprule\noalign{}
\begin{minipage}[b]{\linewidth}\raggedright
Field
\end{minipage} & \begin{minipage}[b]{\linewidth}\raggedright
Type
\end{minipage} & \begin{minipage}[b]{\linewidth}\raggedright
Required
\end{minipage} & \begin{minipage}[b]{\linewidth}\raggedright
Description
\end{minipage} & \begin{minipage}[b]{\linewidth}\raggedright
Example
\end{minipage} \\
\midrule\noalign{}
\endhead
\bottomrule\noalign{}
\endlastfoot
\textbf{\texttt{\textless{}title\textgreater{}}} & Text & ✅ Yes &
Human-readable name of the feed & \texttt{"Tech\ News\ Daily"} \\
\textbf{\texttt{\textless{}link\textgreater{}}} & URL & ✅ Yes & Website
URL associated with the feed &
\texttt{"https://technews.example.com"} \\
\textbf{\texttt{\textless{}description\textgreater{}}} & Text & ✅ Yes &
Brief description of the feed content &
\texttt{"Daily\ technology\ news\ and\ analysis"} \\
\textbf{\texttt{\textless{}language\textgreater{}}} & Code & ❌ Optional
& ISO 639 language code & \texttt{"en-us"}, \texttt{"fr-fr"} \\
\textbf{\texttt{\textless{}copyright\textgreater{}}} & Text & ❌
Optional & Copyright notice for the feed content &
\texttt{"©\ 2025\ TechNews\ Corp"} \\
\textbf{\texttt{\textless{}managingEditor\textgreater{}}} & Email & ❌
Optional & Email address of the content editor &
\texttt{"editor@technews.example.com"} \\
\textbf{\texttt{\textless{}webMaster\textgreater{}}} & Email & ❌
Optional & Email address of technical contact &
\texttt{"webmaster@technews.example.com"} \\
\textbf{\texttt{\textless{}pubDate\textgreater{}}} & RFC 822 & ❌
Optional & Publication date of the feed content &
\texttt{"Fri,\ 10\ Oct\ 2025\ 12:00:00\ GMT"} \\
\textbf{\texttt{\textless{}lastBuildDate\textgreater{}}} & RFC 822 & ❌
Optional & Last modification date of the feed &
\texttt{"Fri,\ 10\ Oct\ 2025\ 14:30:00\ GMT"} \\
\textbf{\texttt{\textless{}category\textgreater{}}} & Text & ❌ Optional
& Content categorization (repeatable) & \texttt{"Technology/News"} \\
\textbf{\texttt{\textless{}generator\textgreater{}}} & Text & ❌
Optional & Software used to generate the feed &
\texttt{"WordPress\ 6.4"} \\
\textbf{\texttt{\textless{}docs\textgreater{}}} & URL & ❌ Optional &
Link to RSS specification &
\texttt{"https://cyber.harvard.edu/rss/rss.html"} \\
\textbf{\texttt{\textless{}cloud\textgreater{}}} & Complex & ❌ Optional
& Cloud notification endpoint for push updates & See WebSub section
below \\
\textbf{\texttt{\textless{}ttl\textgreater{}}} & Integer & ❌ Optional &
Time-to-live in minutes (caching hint) & \texttt{60} (refresh after 60
minutes) \\
\textbf{\texttt{\textless{}image\textgreater{}}} & Complex & ❌ Optional
& Feed logo/branding image & Contains
\texttt{\textless{}url\textgreater{}},
\texttt{\textless{}title\textgreater{}},
\texttt{\textless{}link\textgreater{}} \\
\textbf{\texttt{\textless{}textInput\textgreater{}}} & Complex & ❌
Optional & Search box specification & Rarely used in practice \\
\textbf{\texttt{\textless{}skipHours\textgreater{}}} & List & ❌
Optional & Hours when aggregators should skip updates & \texttt{0-23} \\
\textbf{\texttt{\textless{}skipDays\textgreater{}}} & List & ❌ Optional
& Days when aggregators should skip updates & \texttt{Monday},
\texttt{Tuesday}, etc. \\
\end{longtable}

\paragraph{\texorpdfstring{\textbf{Item-Level Data} (Entry
Metadata)}{Item-Level Data (Entry Metadata)}}\label{item-level-data-entry-metadata}

Item elements represent individual entries (articles, episodes, posts)
within the feed.

\begin{longtable}[]{@{}
  >{\raggedright\arraybackslash}p{(\linewidth - 8\tabcolsep) * \real{0.1556}}
  >{\raggedright\arraybackslash}p{(\linewidth - 8\tabcolsep) * \real{0.1333}}
  >{\raggedright\arraybackslash}p{(\linewidth - 8\tabcolsep) * \real{0.2222}}
  >{\raggedright\arraybackslash}p{(\linewidth - 8\tabcolsep) * \real{0.2889}}
  >{\raggedright\arraybackslash}p{(\linewidth - 8\tabcolsep) * \real{0.2000}}@{}}
\toprule\noalign{}
\begin{minipage}[b]{\linewidth}\raggedright
Field
\end{minipage} & \begin{minipage}[b]{\linewidth}\raggedright
Type
\end{minipage} & \begin{minipage}[b]{\linewidth}\raggedright
Required
\end{minipage} & \begin{minipage}[b]{\linewidth}\raggedright
Description
\end{minipage} & \begin{minipage}[b]{\linewidth}\raggedright
Example
\end{minipage} \\
\midrule\noalign{}
\endhead
\bottomrule\noalign{}
\endlastfoot
\textbf{\texttt{\textless{}title\textgreater{}}} & Text & * & Title of
the item & \texttt{"Breaking:\ New\ AI\ Breakthrough"} \\
\textbf{\texttt{\textless{}link\textgreater{}}} & URL & * & Permanent
URL for the item &
\texttt{"https://technews.example.com/article-123"} \\
\textbf{\texttt{\textless{}description\textgreater{}}} & HTML/Text & * &
Item content or summary & Can contain full HTML content \\
\textbf{\texttt{\textless{}author\textgreater{}}} & Email & ❌ Optional
& Author's email address &
\texttt{"jane.doe@example.com\ (Jane\ Doe)"} \\
\textbf{\texttt{\textless{}category\textgreater{}}} & Text & ❌ Optional
& Item categorization (repeatable) &
\texttt{"Artificial\ Intelligence"} \\
\textbf{\texttt{\textless{}comments\textgreater{}}} & URL & ❌ Optional
& URL to comments page &
\texttt{"https://technews.example.com/article-123\#comments"} \\
\textbf{\texttt{\textless{}enclosure\textgreater{}}} & Complex & ❌
Optional & Attached media file (podcast audio, video) & See table
below \\
\textbf{\texttt{\textless{}guid\textgreater{}}} & Text & ❌ Optional &
Globally unique identifier & \texttt{"article-123"} or permalink URL \\
\textbf{\texttt{\textless{}pubDate\textgreater{}}} & RFC 822 & ❌
Optional & Publication date of the item &
\texttt{"Thu,\ 09\ Oct\ 2025\ 18:45:00\ GMT"} \\
\textbf{\texttt{\textless{}source\textgreater{}}} & Complex & ❌
Optional & Original feed if republished content & Contains
\texttt{\textless{}url\textgreater{}} and
\texttt{\textless{}title\textgreater{}} \\
\end{longtable}

\textbf{Note}: * indicates that \textbf{at least one} of
\texttt{\textless{}title\textgreater{}} or
\texttt{\textless{}description\textgreater{}} must be present.

\paragraph{\texorpdfstring{\textbf{Enclosure Element} (Media
Attachments)}{Enclosure Element (Media Attachments)}}\label{enclosure-element-media-attachments}

The \texttt{\textless{}enclosure\textgreater{}} element enables podcast
and media distribution:

\begin{longtable}[]{@{}
  >{\raggedright\arraybackslash}p{(\linewidth - 8\tabcolsep) * \real{0.2245}}
  >{\raggedright\arraybackslash}p{(\linewidth - 8\tabcolsep) * \real{0.1224}}
  >{\raggedright\arraybackslash}p{(\linewidth - 8\tabcolsep) * \real{0.2041}}
  >{\raggedright\arraybackslash}p{(\linewidth - 8\tabcolsep) * \real{0.2653}}
  >{\raggedright\arraybackslash}p{(\linewidth - 8\tabcolsep) * \real{0.1837}}@{}}
\toprule\noalign{}
\begin{minipage}[b]{\linewidth}\raggedright
Attribute
\end{minipage} & \begin{minipage}[b]{\linewidth}\raggedright
Type
\end{minipage} & \begin{minipage}[b]{\linewidth}\raggedright
Required
\end{minipage} & \begin{minipage}[b]{\linewidth}\raggedright
Description
\end{minipage} & \begin{minipage}[b]{\linewidth}\raggedright
Example
\end{minipage} \\
\midrule\noalign{}
\endhead
\bottomrule\noalign{}
\endlastfoot
\textbf{\texttt{url}} & URL & ✅ Yes & Direct URL to the media file &
\texttt{"https://cdn.example.com/episode42.mp3"} \\
\textbf{\texttt{length}} & Integer & ✅ Yes & File size in bytes &
\texttt{48234567} (48.2 MB) \\
\textbf{\texttt{type}} & MIME & ✅ Yes & Media type &
\texttt{"audio/mpeg"}, \texttt{"video/mp4"} \\
\end{longtable}

\begin{Shaded}
\begin{Highlighting}[]
\NormalTok{\textless{}}\KeywordTok{enclosure}\OtherTok{ url=}\StringTok{"https://cdn.example.com/episode42.mp3"} 
\OtherTok{           length=}\StringTok{"48234567"} 
\OtherTok{           type=}\StringTok{"audio/mpeg"}\NormalTok{/\textgreater{}}
\end{Highlighting}
\end{Shaded}

\paragraph{\texorpdfstring{\textbf{Namespace
Extensions}}{Namespace Extensions}}\label{namespace-extensions}

RSS 2.0 supports XML namespaces for additional metadata. The most common
is the \textbf{iTunes podcast namespace}:

\subparagraph{\texorpdfstring{iTunes Podcast Extensions
(\texttt{xmlns:itunes})}{iTunes Podcast Extensions (xmlns:itunes)}}\label{itunes-podcast-extensions-xmlnsitunes}

\begin{longtable}[]{@{}
  >{\raggedright\arraybackslash}p{(\linewidth - 4\tabcolsep) * \real{0.2903}}
  >{\raggedright\arraybackslash}p{(\linewidth - 4\tabcolsep) * \real{0.4194}}
  >{\raggedright\arraybackslash}p{(\linewidth - 4\tabcolsep) * \real{0.2903}}@{}}
\toprule\noalign{}
\begin{minipage}[b]{\linewidth}\raggedright
Element
\end{minipage} & \begin{minipage}[b]{\linewidth}\raggedright
Description
\end{minipage} & \begin{minipage}[b]{\linewidth}\raggedright
Example
\end{minipage} \\
\midrule\noalign{}
\endhead
\bottomrule\noalign{}
\endlastfoot
\textbf{\texttt{\textless{}itunes:author\textgreater{}}} &
Podcast/episode author & \texttt{"Jane\ Tech"} \\
\textbf{\texttt{\textless{}itunes:subtitle\textgreater{}}} & Short
description & \texttt{"AI\ in\ Healthcare"} \\
\textbf{\texttt{\textless{}itunes:summary\textgreater{}}} & Full
description &
\texttt{"A\ deep\ dive\ into\ medical\ AI\ applications"} \\
\textbf{\texttt{\textless{}itunes:duration\textgreater{}}} & Episode
length & \texttt{"45:30"} (HH:MM:SS or seconds) \\
\textbf{\texttt{\textless{}itunes:image\textgreater{}}} & Artwork URL &
\texttt{\textless{}itunes:image\ href="artwork.jpg"/\textgreater{}} \\
\textbf{\texttt{\textless{}itunes:explicit\textgreater{}}} & Content
rating & \texttt{"true"}, \texttt{"false"}, \texttt{"clean"} \\
\textbf{\texttt{\textless{}itunes:category\textgreater{}}} & Podcast
category &
\texttt{\textless{}itunes:category\ text="Technology"/\textgreater{}} \\
\textbf{\texttt{\textless{}itunes:owner\textgreater{}}} & Publisher
contact & Contains \texttt{\textless{}itunes:name\textgreater{}} and
\texttt{\textless{}itunes:email\textgreater{}} \\
\textbf{\texttt{\textless{}itunes:type\textgreater{}}} & Show type &
\texttt{"episodic"} or \texttt{"serial"} \\
\textbf{\texttt{\textless{}itunes:episode\textgreater{}}} & Episode
number & \texttt{42} \\
\textbf{\texttt{\textless{}itunes:season\textgreater{}}} & Season number
& \texttt{3} \\
\end{longtable}

\begin{center}\rule{0.5\linewidth}{0.5pt}\end{center}

\subsubsection{🔔 How RSS Notifications Are
Received}\label{how-rss-notifications-are-received}

RSS 2.0 primarily uses a \textbf{pull-based model}, with limited support
for push notifications.

\paragraph{\texorpdfstring{\textbf{1. Pull Mechanism (Standard
Approach)}}{1. Pull Mechanism (Standard Approach)}}\label{pull-mechanism-standard-approach}

\textbf{Protocol}: HTTP/HTTPS GET requests

\textbf{Process Flow}:

\begin{verbatim}
┌─────────────┐                                    ┌─────────────┐
│   Client    │                                    │ RSS Server  │
│ (Aggregator)│                                    │             │
└──────┬──────┘                                    └──────┬──────┘
       │                                                  │
       │  1. HTTP GET /feed.xml                          │
       ├─────────────────────────────────────────────────>│
       │                                                  │
       │  2. 200 OK + XML Content                        │
       │<─────────────────────────────────────────────────┤
       │                                                  │
       │  3. Parse XML                                    │
       │  4. Compare <guid> or <pubDate>                  │
       │  5. Download new items                           │
       │                                                  │
       │  6. Wait (based on <ttl> or schedule)           │
       │  ...                                             │
       │  7. HTTP GET /feed.xml (repeat)                 │
       ├─────────────────────────────────────────────────>│
\end{verbatim}

\textbf{Key Characteristics}:

\begin{itemize}
\tightlist
\item
  \textbf{Polling Interval}: Client determines frequency (hourly, daily,
  based on \texttt{\textless{}ttl\textgreater{}})
\item
  \textbf{Change Detection}: Compare
  \texttt{\textless{}lastBuildDate\textgreater{}},
  \texttt{\textless{}pubDate\textgreater{}}, or individual
  \texttt{\textless{}guid\textgreater{}} values
\item
  \textbf{Conditional Requests}: Use HTTP headers
  (\texttt{If-Modified-Since}, \texttt{ETag}) to minimize bandwidth
\item
  \textbf{Caching}: Respect \texttt{\textless{}ttl\textgreater{}}
  (time-to-live) hint to avoid excessive server load
\end{itemize}

\textbf{Advantages}: - ✅ Universal compatibility (works with all RSS
feeds) - ✅ Simple implementation - ✅ Client controls update frequency
- ✅ No additional infrastructure required

\textbf{Disadvantages}: - ❌ Update latency (delay between publication
and discovery) - ❌ Bandwidth waste (polling unchanged feeds) - ❌
Server load (multiple clients polling simultaneously) - ❌ Not real-time

\paragraph{\texorpdfstring{\textbf{2. Push Mechanism (Cloud Element /
RSSCloud)}}{2. Push Mechanism (Cloud Element / RSSCloud)}}\label{push-mechanism-cloud-element-rsscloud}

RSS 2.0 includes an optional \texttt{\textless{}cloud\textgreater{}}
element for push notifications.

\textbf{Protocol}: RSSCloud (proprietary notification system)

\textbf{XML Structure}:

\begin{Shaded}
\begin{Highlighting}[]
\NormalTok{\textless{}}\KeywordTok{cloud}\OtherTok{ domain=}\StringTok{"rpc.example.com"} 
\OtherTok{       port=}\StringTok{"80"} 
\OtherTok{       path=}\StringTok{"/RPC2"} 
\OtherTok{       registerProcedure=}\StringTok{"pleaseNotify"} 
\OtherTok{       protocol=}\StringTok{"xml{-}rpc"}\NormalTok{/\textgreater{}}
\end{Highlighting}
\end{Shaded}

\textbf{Attribute Meanings}:

\begin{longtable}[]{@{}
  >{\raggedright\arraybackslash}p{(\linewidth - 4\tabcolsep) * \real{0.3333}}
  >{\raggedright\arraybackslash}p{(\linewidth - 4\tabcolsep) * \real{0.3939}}
  >{\raggedright\arraybackslash}p{(\linewidth - 4\tabcolsep) * \real{0.2727}}@{}}
\toprule\noalign{}
\begin{minipage}[b]{\linewidth}\raggedright
Attribute
\end{minipage} & \begin{minipage}[b]{\linewidth}\raggedright
Description
\end{minipage} & \begin{minipage}[b]{\linewidth}\raggedright
Example
\end{minipage} \\
\midrule\noalign{}
\endhead
\bottomrule\noalign{}
\endlastfoot
\textbf{\texttt{domain}} & Notification server hostname &
\texttt{"rpc.example.com"} \\
\textbf{\texttt{port}} & Server port & \texttt{80}, \texttt{443} \\
\textbf{\texttt{path}} & Endpoint path & \texttt{"/RPC2"} \\
\textbf{\texttt{registerProcedure}} & Registration method name &
\texttt{"pleaseNotify"} \\
\textbf{\texttt{protocol}} & Notification protocol & \texttt{"xml-rpc"},
\texttt{"soap"}, \texttt{"http-post"} \\
\end{longtable}

\textbf{Process Flow}:

\begin{verbatim}
┌─────────────┐         ┌─────────────┐         ┌─────────────┐
│   Client    │         │Cloud Server │         │ RSS Server  │
└──────┬──────┘         └──────┬──────┘         └──────┬──────┘
       │                       │                        │
       │ 1. Register for       │                        │
       │    notifications      │                        │
       ├──────────────────────>│                        │
       │                       │                        │
       │                       │  2. Content updated    │
       │                       │<───────────────────────┤
       │                       │                        │
       │ 3. Notification       │                        │
       │    (feed changed)     │                        │
       │<──────────────────────┤                        │
       │                       │                        │
       │ 4. HTTP GET /feed.xml │                        │
       ├───────────────────────┼───────────────────────>│
       │                       │                        │
       │ 5. 200 OK + New Content                        │
       │<───────────────────────┼────────────────────────┤
\end{verbatim}

\textbf{Advantages}: - ✅ Immediate notification of updates - ✅ Reduced
polling overhead - ✅ More efficient bandwidth usage

\textbf{Disadvantages}: - ❌ Extremely rare in practice (almost no
implementations) - ❌ Not standardized (multiple competing protocols) -
❌ Complex infrastructure requirements - ❌ Largely superseded by WebSub

\paragraph{\texorpdfstring{\textbf{3. Push Mechanism (WebSub
Integration)}}{3. Push Mechanism (WebSub Integration)}}\label{push-mechanism-websub-integration}

Modern RSS feeds often integrate \textbf{WebSub} (formerly PubSubHubbub)
for real-time notifications.

\textbf{Protocol}: WebSub (W3C Recommendation)

\textbf{Discovery via HTTP Link Headers}:

\begin{Shaded}
\begin{Highlighting}[]
\NormalTok{HTTP/1.1 200 OK}
\NormalTok{Link: \textless{}https://hub.example.com/\textgreater{}; rel="hub"}
\NormalTok{Link: \textless{}https://publisher.example.com/feed.xml\textgreater{}; rel="self"}
\end{Highlighting}
\end{Shaded}

\textbf{Or via RSS XML Elements}:

\begin{Shaded}
\begin{Highlighting}[]
\NormalTok{\textless{}}\KeywordTok{rss}\OtherTok{ version=}\StringTok{"2.0"}\OtherTok{ xmlns:atom=}\StringTok{"http://www.w3.org/2005/Atom"}\NormalTok{\textgreater{}}
\NormalTok{  \textless{}}\KeywordTok{channel}\NormalTok{\textgreater{}}
\NormalTok{    \textless{}}\KeywordTok{atom:link}\OtherTok{ href=}\StringTok{"https://hub.example.com/"}\OtherTok{ rel=}\StringTok{"hub"}\NormalTok{/\textgreater{}}
\NormalTok{    \textless{}}\KeywordTok{atom:link}\OtherTok{ href=}\StringTok{"https://publisher.example.com/feed.xml"}\OtherTok{ rel=}\StringTok{"self"}\NormalTok{/\textgreater{}}
    \CommentTok{\textless{}!{-}{-} Feed content {-}{-}\textgreater{}}
\NormalTok{  \textless{}/}\KeywordTok{channel}\NormalTok{\textgreater{}}
\NormalTok{\textless{}/}\KeywordTok{rss}\NormalTok{\textgreater{}}
\end{Highlighting}
\end{Shaded}

\textbf{Process Flow}:

\begin{verbatim}
┌─────────────┐         ┌─────────────┐         ┌─────────────┐
│ Subscriber  │         │  WebSub Hub │         │  Publisher  │
└──────┬──────┘         └──────┬──────┘         └──────┬──────┘
       │                       │                        │
       │ 1. Subscribe to topic │                        │
       ├──────────────────────>│                        │
       │                       │                        │
       │ 2. Verify intent      │                        │
       │<──────────────────────┤                        │
       │                       │                        │
       │ 3. Confirm            │                        │
       ├──────────────────────>│                        │
       │                       │                        │
       │                       │  4. Publish update     │
       │                       │<───────────────────────┤
       │                       │                        │
       │ 5. Content push       │                        │
       │    (full feed XML)    │                        │
       │<──────────────────────┤                        │
\end{verbatim}

\textbf{Key Operations}:

\begin{enumerate}
\def\labelenumi{\arabic{enumi}.}
\tightlist
\item
  \textbf{Discovery}: Client finds hub URL in feed or HTTP headers
\item
  \textbf{Subscription}: Client sends POST to hub with callback URL and
  topic
\item
  \textbf{Verification}: Hub confirms subscription via GET to callback
  URL
\item
  \textbf{Publishing}: Publisher notifies hub when content changes
\item
  \textbf{Distribution}: Hub pushes updated feed to all subscribers
\end{enumerate}

\textbf{Advantages}: - ✅ Real-time updates (sub-second latency
possible) - ✅ Standardized W3C protocol - ✅ Decentralized architecture
- ✅ Efficient bandwidth usage

\textbf{Disadvantages}: - ❌ Limited adoption in RSS ecosystem (more
common with Atom) - ❌ Requires public callback URL (challenging for
mobile/desktop apps) - ❌ Additional infrastructure complexity - ❌
Potential reliability issues if hub is unavailable

\begin{center}\rule{0.5\linewidth}{0.5pt}\end{center}

\subsubsection{📊 RSS Data Summary}\label{rss-data-summary}

\textbf{Data Richness}: \textbf{Moderate to High} - Extensible via
namespaces (iTunes, Dublin Core, Media RSS) - Basic metadata sufficient
for most use cases - Podcast-specific extensions widely supported

\textbf{Notification Model}: \textbf{Primarily Pull, Optional Push} -
Default: HTTP polling (pull) - Legacy: RSSCloud (rarely implemented) -
Modern: WebSub integration (growing adoption)

\begin{center}\rule{0.5\linewidth}{0.5pt}\end{center}

\subsection{⚛️ Atom Specification
Analysis}\label{atom-specification-analysis}

\subsubsection{Overview}\label{overview-1}

\textbf{Atom} is an IETF-standardized syndication format designed to
address ambiguities and limitations in RSS. Published as RFC 4287 in
2005, Atom emphasizes formal specification, validation, and protocol
clarity.

\begin{quote}
📖 \textbf{Specification}: Atom is defined in
\href{https://tools.ietf.org/html/rfc4287}{RFC 4287} and the publishing
protocol in \href{https://tools.ietf.org/html/rfc5023}{RFC 5023}.
\end{quote}

\begin{center}\rule{0.5\linewidth}{0.5pt}\end{center}

\subsubsection{📦 Data Available from Atom
Notifications}\label{data-available-from-atom-notifications}

Atom provides a more structured and formally defined data model than
RSS.

\paragraph{\texorpdfstring{\textbf{Feed-Level Data} (Feed
Metadata)}{Feed-Level Data (Feed Metadata)}}\label{feed-level-data-feed-metadata}

Feed elements describe the overall feed container.

\begin{longtable}[]{@{}
  >{\raggedright\arraybackslash}p{(\linewidth - 8\tabcolsep) * \real{0.1915}}
  >{\raggedright\arraybackslash}p{(\linewidth - 8\tabcolsep) * \real{0.1277}}
  >{\raggedright\arraybackslash}p{(\linewidth - 8\tabcolsep) * \real{0.2128}}
  >{\raggedright\arraybackslash}p{(\linewidth - 8\tabcolsep) * \real{0.2766}}
  >{\raggedright\arraybackslash}p{(\linewidth - 8\tabcolsep) * \real{0.1915}}@{}}
\toprule\noalign{}
\begin{minipage}[b]{\linewidth}\raggedright
Element
\end{minipage} & \begin{minipage}[b]{\linewidth}\raggedright
Type
\end{minipage} & \begin{minipage}[b]{\linewidth}\raggedright
Required
\end{minipage} & \begin{minipage}[b]{\linewidth}\raggedright
Description
\end{minipage} & \begin{minipage}[b]{\linewidth}\raggedright
Example
\end{minipage} \\
\midrule\noalign{}
\endhead
\bottomrule\noalign{}
\endlastfoot
\textbf{\texttt{\textless{}id\textgreater{}}} & IRI & ✅ Yes &
Permanent, globally unique feed identifier (IRI) &
\texttt{"https://example.com/feeds/blog"} \\
\textbf{\texttt{\textless{}title\textgreater{}}} & Text & ✅ Yes &
Human-readable feed title & \texttt{"Tech\ Insights\ Blog"} \\
\textbf{\texttt{\textless{}updated\textgreater{}}} & RFC 3339 & ✅ Yes &
Last modification timestamp & \texttt{"2025-10-10T14:30:00Z"} \\
\textbf{\texttt{\textless{}author\textgreater{}}} & Person & ❌
Optional* & Feed author information & See Person Construct below \\
\textbf{\texttt{\textless{}link\textgreater{}}} & Link & ❌ Optional &
Related resources (website, self-reference) & See Link Construct
below \\
\textbf{\texttt{\textless{}category\textgreater{}}} & Category & ❌
Optional & Feed categorization (repeatable) & See Category Construct
below \\
\textbf{\texttt{\textless{}contributor\textgreater{}}} & Person & ❌
Optional & Additional contributors & See Person Construct below \\
\textbf{\texttt{\textless{}generator\textgreater{}}} & Text & ❌
Optional & Software generating the feed & \texttt{"WordPress\ 6.4"} with
optional \texttt{uri} and \texttt{version} \\
\textbf{\texttt{\textless{}icon\textgreater{}}} & IRI & ❌ Optional &
Small icon (square, recommended 1:1 aspect) &
\texttt{"https://example.com/icon.png"} \\
\textbf{\texttt{\textless{}logo\textgreater{}}} & IRI & ❌ Optional &
Larger logo (recommended 2:1 aspect) &
\texttt{"https://example.com/logo.png"} \\
\textbf{\texttt{\textless{}rights\textgreater{}}} & Text & ❌ Optional &
Copyright/licensing information &
\texttt{"©\ 2025\ Example\ Corp.\ All\ rights\ reserved."} \\
\textbf{\texttt{\textless{}subtitle\textgreater{}}} & Text & ❌ Optional
& Feed description/tagline &
\texttt{"Exploring\ technology\ trends\ and\ insights"} \\
\end{longtable}

\textbf{Note}: * If an entry lacks an
\texttt{\textless{}author\textgreater{}} element, the feed MUST have an
\texttt{\textless{}author\textgreater{}} element.

\paragraph{\texorpdfstring{\textbf{Entry-Level Data} (Individual Item
Metadata)}{Entry-Level Data (Individual Item Metadata)}}\label{entry-level-data-individual-item-metadata}

Entry elements represent individual items within the feed.

\begin{longtable}[]{@{}
  >{\raggedright\arraybackslash}p{(\linewidth - 8\tabcolsep) * \real{0.1915}}
  >{\raggedright\arraybackslash}p{(\linewidth - 8\tabcolsep) * \real{0.1277}}
  >{\raggedright\arraybackslash}p{(\linewidth - 8\tabcolsep) * \real{0.2128}}
  >{\raggedright\arraybackslash}p{(\linewidth - 8\tabcolsep) * \real{0.2766}}
  >{\raggedright\arraybackslash}p{(\linewidth - 8\tabcolsep) * \real{0.1915}}@{}}
\toprule\noalign{}
\begin{minipage}[b]{\linewidth}\raggedright
Element
\end{minipage} & \begin{minipage}[b]{\linewidth}\raggedright
Type
\end{minipage} & \begin{minipage}[b]{\linewidth}\raggedright
Required
\end{minipage} & \begin{minipage}[b]{\linewidth}\raggedright
Description
\end{minipage} & \begin{minipage}[b]{\linewidth}\raggedright
Example
\end{minipage} \\
\midrule\noalign{}
\endhead
\bottomrule\noalign{}
\endlastfoot
\textbf{\texttt{\textless{}id\textgreater{}}} & IRI & ✅ Yes &
Permanent, globally unique entry identifier &
\texttt{"https://example.com/posts/2025/10/article-123"} \\
\textbf{\texttt{\textless{}title\textgreater{}}} & Text & ✅ Yes &
Human-readable entry title &
\texttt{"Understanding\ Quantum\ Computing"} \\
\textbf{\texttt{\textless{}updated\textgreater{}}} & RFC 3339 & ✅ Yes &
Last modification timestamp & \texttt{"2025-10-09T18:45:00Z"} \\
\textbf{\texttt{\textless{}author\textgreater{}}} & Person & ❌
Optional* & Entry author information & See Person Construct below \\
\textbf{\texttt{\textless{}content\textgreater{}}} & Content & ❌
Optional** & Full or partial entry content & See Content Construct
below \\
\textbf{\texttt{\textless{}link\textgreater{}}} & Link & ❌ Optional** &
Related resources (alternate, enclosure) & See Link Construct below \\
\textbf{\texttt{\textless{}summary\textgreater{}}} & Text & ❌
Optional** & Brief entry summary or excerpt &
\texttt{"An\ introduction\ to\ quantum\ computing\ principles"} \\
\textbf{\texttt{\textless{}category\textgreater{}}} & Category & ❌
Optional & Entry categorization (repeatable) & See Category Construct
below \\
\textbf{\texttt{\textless{}contributor\textgreater{}}} & Person & ❌
Optional & Additional contributors & See Person Construct below \\
\textbf{\texttt{\textless{}published\textgreater{}}} & RFC 3339 & ❌
Optional & Original publication timestamp &
\texttt{"2025-10-09T10:00:00Z"} \\
\textbf{\texttt{\textless{}rights\textgreater{}}} & Text & ❌ Optional &
Copyright/licensing for entry & \texttt{"CC\ BY-SA\ 4.0"} \\
\textbf{\texttt{\textless{}source\textgreater{}}} & Feed & ❌ Optional &
Original feed metadata if aggregated & Contains feed-level elements \\
\end{longtable}

\textbf{Notes}: - * If entry lacks
\texttt{\textless{}author\textgreater{}}, feed MUST have
\texttt{\textless{}author\textgreater{}} - ** Entry MUST contain at
least one \texttt{\textless{}link\ rel="alternate"\textgreater{}} or
\texttt{\textless{}content\textgreater{}}

\paragraph{\texorpdfstring{\textbf{Atom Constructs} (Complex Data
Types)}{Atom Constructs (Complex Data Types)}}\label{atom-constructs-complex-data-types}

Atom uses reusable constructs for structured data:

\subparagraph{\texorpdfstring{\textbf{Person Construct}
(\texttt{\textless{}author\textgreater{}},
\texttt{\textless{}contributor\textgreater{}})}{Person Construct (\textless author\textgreater, \textless contributor\textgreater)}}\label{person-construct-author-contributor}

\begin{Shaded}
\begin{Highlighting}[]
\NormalTok{\textless{}}\KeywordTok{author}\NormalTok{\textgreater{}}
\NormalTok{  \textless{}}\KeywordTok{name}\NormalTok{\textgreater{}Jane Smith\textless{}/}\KeywordTok{name}\NormalTok{\textgreater{}}
\NormalTok{  \textless{}}\KeywordTok{uri}\NormalTok{\textgreater{}https://janesmith.com\textless{}/}\KeywordTok{uri}\NormalTok{\textgreater{}}
\NormalTok{  \textless{}}\KeywordTok{email}\NormalTok{\textgreater{}jane@example.com\textless{}/}\KeywordTok{email}\NormalTok{\textgreater{}}
\NormalTok{\textless{}/}\KeywordTok{author}\NormalTok{\textgreater{}}
\end{Highlighting}
\end{Shaded}

\begin{longtable}[]{@{}
  >{\raggedright\arraybackslash}p{(\linewidth - 4\tabcolsep) * \real{0.3611}}
  >{\raggedright\arraybackslash}p{(\linewidth - 4\tabcolsep) * \real{0.2778}}
  >{\raggedright\arraybackslash}p{(\linewidth - 4\tabcolsep) * \real{0.3611}}@{}}
\toprule\noalign{}
\begin{minipage}[b]{\linewidth}\raggedright
Sub-element
\end{minipage} & \begin{minipage}[b]{\linewidth}\raggedright
Required
\end{minipage} & \begin{minipage}[b]{\linewidth}\raggedright
Description
\end{minipage} \\
\midrule\noalign{}
\endhead
\bottomrule\noalign{}
\endlastfoot
\textbf{\texttt{\textless{}name\textgreater{}}} & ✅ Yes & Person's
name \\
\textbf{\texttt{\textless{}uri\textgreater{}}} & ❌ Optional & IRI
associated with person (homepage, profile) \\
\textbf{\texttt{\textless{}email\textgreater{}}} & ❌ Optional & Email
address \\
\end{longtable}

\subparagraph{\texorpdfstring{\textbf{Link Construct}
(\texttt{\textless{}link\textgreater{}})}{Link Construct (\textless link\textgreater)}}\label{link-construct-link}

\begin{Shaded}
\begin{Highlighting}[]
\NormalTok{\textless{}}\KeywordTok{link}\OtherTok{ rel=}\StringTok{"alternate"}\OtherTok{ type=}\StringTok{"text/html"}\OtherTok{ href=}\StringTok{"https://example.com/post"}\NormalTok{/\textgreater{}}
\NormalTok{\textless{}}\KeywordTok{link}\OtherTok{ rel=}\StringTok{"enclosure"}\OtherTok{ type=}\StringTok{"audio/mpeg"}\OtherTok{ href=}\StringTok{"https://cdn.example.com/audio.mp3"}\OtherTok{ length=}\StringTok{"48234567"}\NormalTok{/\textgreater{}}
\NormalTok{\textless{}}\KeywordTok{link}\OtherTok{ rel=}\StringTok{"self"}\OtherTok{ href=}\StringTok{"https://example.com/feed.xml"}\NormalTok{/\textgreater{}}
\end{Highlighting}
\end{Shaded}

\begin{longtable}[]{@{}
  >{\raggedright\arraybackslash}p{(\linewidth - 6\tabcolsep) * \real{0.2558}}
  >{\raggedright\arraybackslash}p{(\linewidth - 6\tabcolsep) * \real{0.2326}}
  >{\raggedright\arraybackslash}p{(\linewidth - 6\tabcolsep) * \real{0.3023}}
  >{\raggedright\arraybackslash}p{(\linewidth - 6\tabcolsep) * \real{0.2093}}@{}}
\toprule\noalign{}
\begin{minipage}[b]{\linewidth}\raggedright
Attribute
\end{minipage} & \begin{minipage}[b]{\linewidth}\raggedright
Required
\end{minipage} & \begin{minipage}[b]{\linewidth}\raggedright
Description
\end{minipage} & \begin{minipage}[b]{\linewidth}\raggedright
Example
\end{minipage} \\
\midrule\noalign{}
\endhead
\bottomrule\noalign{}
\endlastfoot
\textbf{\texttt{href}} & ✅ Yes & IRI reference &
\texttt{"https://example.com/post"} \\
\textbf{\texttt{rel}} & ❌ Optional & Link relationship type &
\texttt{"alternate"}, \texttt{"enclosure"}, \texttt{"self"},
\texttt{"related"} \\
\textbf{\texttt{type}} & ❌ Optional & MIME media type &
\texttt{"text/html"}, \texttt{"audio/mpeg"} \\
\textbf{\texttt{hreflang}} & ❌ Optional & Language of linked resource &
\texttt{"en-US"}, \texttt{"fr-FR"} \\
\textbf{\texttt{title}} & ❌ Optional & Human-readable title &
\texttt{"Read\ full\ article"} \\
\textbf{\texttt{length}} & ❌ Optional & Size in bytes (for enclosures)
& \texttt{48234567} \\
\end{longtable}

\textbf{Common \texttt{rel} Values}:

\begin{itemize}
\tightlist
\item
  \textbf{\texttt{alternate}}: HTML version of the entry/feed
\item
  \textbf{\texttt{enclosure}}: Related media file (podcast audio,
  attachments)
\item
  \textbf{\texttt{self}}: The feed's own URL
\item
  \textbf{\texttt{related}}: Related resource
\item
  \textbf{\texttt{via}}: Source of the information
\item
  \textbf{\texttt{hub}}: WebSub hub URL (for push notifications)
\end{itemize}

\subparagraph{\texorpdfstring{\textbf{Category Construct}
(\texttt{\textless{}category\textgreater{}})}{Category Construct (\textless category\textgreater)}}\label{category-construct-category}

\begin{Shaded}
\begin{Highlighting}[]
\NormalTok{\textless{}}\KeywordTok{category}\OtherTok{ term=}\StringTok{"technology"}\OtherTok{ scheme=}\StringTok{"http://example.com/categories"}\OtherTok{ label=}\StringTok{"Technology"}\NormalTok{/\textgreater{}}
\end{Highlighting}
\end{Shaded}

\begin{longtable}[]{@{}
  >{\raggedright\arraybackslash}p{(\linewidth - 6\tabcolsep) * \real{0.2558}}
  >{\raggedright\arraybackslash}p{(\linewidth - 6\tabcolsep) * \real{0.2326}}
  >{\raggedright\arraybackslash}p{(\linewidth - 6\tabcolsep) * \real{0.3023}}
  >{\raggedright\arraybackslash}p{(\linewidth - 6\tabcolsep) * \real{0.2093}}@{}}
\toprule\noalign{}
\begin{minipage}[b]{\linewidth}\raggedright
Attribute
\end{minipage} & \begin{minipage}[b]{\linewidth}\raggedright
Required
\end{minipage} & \begin{minipage}[b]{\linewidth}\raggedright
Description
\end{minipage} & \begin{minipage}[b]{\linewidth}\raggedright
Example
\end{minipage} \\
\midrule\noalign{}
\endhead
\bottomrule\noalign{}
\endlastfoot
\textbf{\texttt{term}} & ✅ Yes & Category identifier &
\texttt{"technology"} \\
\textbf{\texttt{scheme}} & ❌ Optional & Categorization scheme IRI &
\texttt{"http://example.com/categories"} \\
\textbf{\texttt{label}} & ❌ Optional & Human-readable label &
\texttt{"Technology"} \\
\end{longtable}

\subparagraph{\texorpdfstring{\textbf{Content Construct}
(\texttt{\textless{}content\textgreater{}})}{Content Construct (\textless content\textgreater)}}\label{content-construct-content}

\begin{Shaded}
\begin{Highlighting}[]
\CommentTok{\textless{}!{-}{-} Text content {-}{-}\textgreater{}}
\NormalTok{\textless{}}\KeywordTok{content}\OtherTok{ type=}\StringTok{"text"}\NormalTok{\textgreater{}This is plain text content.\textless{}/}\KeywordTok{content}\NormalTok{\textgreater{}}

\CommentTok{\textless{}!{-}{-} HTML content {-}{-}\textgreater{}}
\NormalTok{\textless{}}\KeywordTok{content}\OtherTok{ type=}\StringTok{"html"}\NormalTok{\textgreater{}}\DecValTok{\&lt;}\NormalTok{p}\DecValTok{\&gt;}\NormalTok{This is }\DecValTok{\&lt;}\NormalTok{strong}\DecValTok{\&gt;}\NormalTok{HTML}\DecValTok{\&lt;}\NormalTok{/strong}\DecValTok{\&gt;}\NormalTok{ content.}\DecValTok{\&lt;}\NormalTok{/p}\DecValTok{\&gt;}\NormalTok{\textless{}/}\KeywordTok{content}\NormalTok{\textgreater{}}

\CommentTok{\textless{}!{-}{-} XHTML content {-}{-}\textgreater{}}
\NormalTok{\textless{}}\KeywordTok{content}\OtherTok{ type=}\StringTok{"xhtml"}\NormalTok{\textgreater{}}
\NormalTok{  \textless{}}\KeywordTok{div}\OtherTok{ xmlns=}\StringTok{"http://www.w3.org/1999/xhtml"}\NormalTok{\textgreater{}}
\NormalTok{    \textless{}}\KeywordTok{p}\NormalTok{\textgreater{}This is \textless{}}\KeywordTok{strong}\NormalTok{\textgreater{}XHTML\textless{}/}\KeywordTok{strong}\NormalTok{\textgreater{} content.\textless{}/}\KeywordTok{p}\NormalTok{\textgreater{}}
\NormalTok{  \textless{}/}\KeywordTok{div}\NormalTok{\textgreater{}}
\NormalTok{\textless{}/}\KeywordTok{content}\NormalTok{\textgreater{}}

\CommentTok{\textless{}!{-}{-} External content {-}{-}\textgreater{}}
\NormalTok{\textless{}}\KeywordTok{content}\OtherTok{ type=}\StringTok{"audio/mpeg"}\OtherTok{ src=}\StringTok{"https://example.com/audio.mp3"}\NormalTok{/\textgreater{}}
\end{Highlighting}
\end{Shaded}

\begin{longtable}[]{@{}
  >{\raggedright\arraybackslash}p{(\linewidth - 4\tabcolsep) * \real{0.3438}}
  >{\raggedright\arraybackslash}p{(\linewidth - 4\tabcolsep) * \real{0.4062}}
  >{\raggedright\arraybackslash}p{(\linewidth - 4\tabcolsep) * \real{0.2500}}@{}}
\toprule\noalign{}
\begin{minipage}[b]{\linewidth}\raggedright
Attribute
\end{minipage} & \begin{minipage}[b]{\linewidth}\raggedright
Description
\end{minipage} & \begin{minipage}[b]{\linewidth}\raggedright
Values
\end{minipage} \\
\midrule\noalign{}
\endhead
\bottomrule\noalign{}
\endlastfoot
\textbf{\texttt{type}} & Content media type & \texttt{"text"},
\texttt{"html"}, \texttt{"xhtml"}, or MIME type \\
\textbf{\texttt{src}} & External content IRI & Used for out-of-line
content \\
\end{longtable}

\textbf{Content Type Handling}:

\begin{itemize}
\tightlist
\item
  \textbf{\texttt{text}}: Plain text (no markup)
\item
  \textbf{\texttt{html}}: HTML markup (escaped)
\item
  \textbf{\texttt{xhtml}}: XHTML markup (inline XML)
\item
  \textbf{MIME type}: Binary content via \texttt{src} attribute
\end{itemize}

\begin{center}\rule{0.5\linewidth}{0.5pt}\end{center}

\subsubsection{🔔 How Atom Notifications Are
Received}\label{how-atom-notifications-are-received}

Atom supports both pull and push mechanisms, with stronger emphasis on
push via WebSub.

\paragraph{\texorpdfstring{\textbf{1. Pull Mechanism (Standard
HTTP)}}{1. Pull Mechanism (Standard HTTP)}}\label{pull-mechanism-standard-http}

\textbf{Protocol}: HTTP/HTTPS GET requests

\textbf{Process Flow}:

\begin{verbatim}
┌─────────────┐                                    ┌─────────────┐
│   Client    │                                    │ Atom Server │
│ (Aggregator)│                                    │             │
└──────┬──────┘                                    └──────┬──────┘
       │                                                  │
       │  1. HTTP GET /feed.xml                          │
       ├─────────────────────────────────────────────────>│
       │                                                  │
       │  2. 200 OK + Atom XML                           │
       │     Link: <https://hub.com/>; rel="hub"         │
       │<─────────────────────────────────────────────────┤
       │                                                  │
       │  3. Parse Atom XML                              │
       │  4. Compare <updated> or <id> timestamps        │
       │  5. Download new entries                        │
       │                                                  │
       │  6. Wait (based on cache headers or schedule)   │
       │  ...                                             │
       │  7. HTTP GET /feed.xml (repeat)                 │
       ├─────────────────────────────────────────────────>│
\end{verbatim}

\textbf{Key Characteristics}:

\begin{itemize}
\tightlist
\item
  \textbf{Change Detection}: Compare
  \texttt{\textless{}updated\textgreater{}} timestamps at feed and entry
  level
\item
  \textbf{Unique Identifiers}: Use \texttt{\textless{}id\textgreater{}}
  elements (permanent IRIs) to track entries
\item
  \textbf{HTTP Headers}: Support \texttt{ETag}, \texttt{Last-Modified},
  \texttt{If-Modified-Since}, \texttt{If-None-Match}
\item
  \textbf{Caching}: Respect HTTP cache-control headers
\end{itemize}

\textbf{Advantages}: - ✅ Universal compatibility - ✅ Simple
implementation - ✅ Well-defined timestamp semantics

\textbf{Disadvantages}: - ❌ Update latency - ❌ Bandwidth overhead for
unchanged content - ❌ Server load from polling

\paragraph{\texorpdfstring{\textbf{2. Push Mechanism (WebSub -
Recommended)}}{2. Push Mechanism (WebSub - Recommended)}}\label{push-mechanism-websub---recommended}

Atom has strong integration with \textbf{WebSub} (W3C Recommendation),
making it the preferred protocol for push notifications.

\textbf{Protocol}: WebSub (formerly PubSubHubbub)

\textbf{Discovery in Atom Feed}:

\begin{Shaded}
\begin{Highlighting}[]
\FunctionTok{\textless{}?xml}\OtherTok{ version=}\StringTok{"1.0"}\OtherTok{ encoding=}\StringTok{"utf{-}8"}\FunctionTok{?\textgreater{}}
\NormalTok{\textless{}}\KeywordTok{feed}\OtherTok{ xmlns=}\StringTok{"http://www.w3.org/2005/Atom"}\NormalTok{\textgreater{}}
\NormalTok{  \textless{}}\KeywordTok{id}\NormalTok{\textgreater{}https://example.com/feed\textless{}/}\KeywordTok{id}\NormalTok{\textgreater{}}
\NormalTok{  \textless{}}\KeywordTok{title}\NormalTok{\textgreater{}Tech Blog\textless{}/}\KeywordTok{title}\NormalTok{\textgreater{}}
\NormalTok{  \textless{}}\KeywordTok{updated}\NormalTok{\textgreater{}2025{-}10{-}10T14:30:00Z\textless{}/}\KeywordTok{updated}\NormalTok{\textgreater{}}
  
  \CommentTok{\textless{}!{-}{-} WebSub Hub Discovery {-}{-}\textgreater{}}
\NormalTok{  \textless{}}\KeywordTok{link}\OtherTok{ rel=}\StringTok{"hub"}\OtherTok{ href=}\StringTok{"https://pubsubhubbub.appspot.com/"}\NormalTok{/\textgreater{}}
\NormalTok{  \textless{}}\KeywordTok{link}\OtherTok{ rel=}\StringTok{"self"}\OtherTok{ href=}\StringTok{"https://example.com/feed.xml"}\NormalTok{/\textgreater{}}
  
  \CommentTok{\textless{}!{-}{-} Feed content {-}{-}\textgreater{}}
\NormalTok{\textless{}/}\KeywordTok{feed}\NormalTok{\textgreater{}}
\end{Highlighting}
\end{Shaded}

\textbf{Or via HTTP Headers}:

\begin{Shaded}
\begin{Highlighting}[]
\NormalTok{HTTP/1.1 200 OK}
\NormalTok{Content{-}Type: application/atom+xml}
\NormalTok{Link: \textless{}https://pubsubhubbub.appspot.com/\textgreater{}; rel="hub"}
\NormalTok{Link: \textless{}https://example.com/feed.xml\textgreater{}; rel="self"}
\end{Highlighting}
\end{Shaded}

\textbf{Process Flow}:

\begin{verbatim}
┌─────────────┐         ┌─────────────┐         ┌─────────────┐
│ Subscriber  │         │  WebSub Hub │         │  Publisher  │
└──────┬──────┘         └──────┬──────┘         └──────┬──────┘
       │                       │                        │
       │ 1. POST Subscribe     │                        │
       │    topic: feed URL    │                        │
       │    callback: https:// │                        │
       ├──────────────────────>│                        │
       │                       │                        │
       │ 2. GET Verify Intent  │                        │
       │    ?hub.challenge=... │                        │
       │<──────────────────────┤                        │
       │                       │                        │
       │ 3. 200 OK             │                        │
       │    (echo challenge)   │                        │
       ├──────────────────────>│                        │
       │                       │                        │
       │                       │  4. POST Publish       │
       │                       │<───────────────────────┤
       │                       │                        │
       │ 5. POST Content       │                        │
       │    (full Atom feed)   │                        │
       │<──────────────────────┤                        │
       │                       │                        │
       │ 6. 200 OK             │                        │
       ├──────────────────────>│                        │
\end{verbatim}

\textbf{Subscription Request}:

\begin{Shaded}
\begin{Highlighting}[]
\NormalTok{POST /subscribe HTTP/1.1}
\NormalTok{Host: pubsubhubbub.appspot.com}
\NormalTok{Content{-}Type: application/x{-}www{-}form{-}urlencoded}

\NormalTok{hub.mode=subscribe}
\NormalTok{\&hub.topic=https://example.com/feed.xml}
\NormalTok{\&hub.callback=https://subscriber.example.com/webhook}
\NormalTok{\&hub.lease\_seconds=864000}
\NormalTok{\&hub.secret=my\_secret\_key}
\end{Highlighting}
\end{Shaded}

\textbf{Parameters}:

\begin{longtable}[]{@{}
  >{\raggedright\arraybackslash}p{(\linewidth - 4\tabcolsep) * \real{0.3235}}
  >{\raggedright\arraybackslash}p{(\linewidth - 4\tabcolsep) * \real{0.2941}}
  >{\raggedright\arraybackslash}p{(\linewidth - 4\tabcolsep) * \real{0.3824}}@{}}
\toprule\noalign{}
\begin{minipage}[b]{\linewidth}\raggedright
Parameter
\end{minipage} & \begin{minipage}[b]{\linewidth}\raggedright
Required
\end{minipage} & \begin{minipage}[b]{\linewidth}\raggedright
Description
\end{minipage} \\
\midrule\noalign{}
\endhead
\bottomrule\noalign{}
\endlastfoot
\textbf{\texttt{hub.mode}} & ✅ Yes & \texttt{"subscribe"} or
\texttt{"unsubscribe"} \\
\textbf{\texttt{hub.topic}} & ✅ Yes & Feed URL to subscribe to \\
\textbf{\texttt{hub.callback}} & ✅ Yes & Subscriber's webhook URL \\
\textbf{\texttt{hub.lease\_seconds}} & ❌ Optional & Subscription
duration (default: hub-specific) \\
\textbf{\texttt{hub.secret}} & ❌ Optional & Shared secret for HMAC
verification \\
\end{longtable}

\textbf{Intent Verification}:

The hub verifies the subscription by sending a GET request to the
callback URL:

\begin{Shaded}
\begin{Highlighting}[]
\NormalTok{GET /webhook?hub.mode=subscribe}
\NormalTok{            \&hub.topic=https://example.com/feed.xml}
\NormalTok{            \&hub.challenge=random\_string\_12345}
\NormalTok{            \&hub.lease\_seconds=864000 HTTP/1.1}
\NormalTok{Host: subscriber.example.com}
\end{Highlighting}
\end{Shaded}

Subscriber must respond with:

\begin{Shaded}
\begin{Highlighting}[]
\NormalTok{HTTP/1.1 200 OK}
\NormalTok{Content{-}Type: text/plain}

\NormalTok{random\_string\_12345}
\end{Highlighting}
\end{Shaded}

\textbf{Content Distribution}:

When the publisher updates the feed, the hub pushes the full Atom feed
to all subscribers:

\begin{Shaded}
\begin{Highlighting}[]
\NormalTok{POST /webhook HTTP/1.1}
\NormalTok{Host: subscriber.example.com}
\NormalTok{Content{-}Type: application/atom+xml}
\NormalTok{X{-}Hub{-}Signature: sha256=abc123...}

\NormalTok{\textless{}?xml version="1.0" encoding="utf{-}8"?\textgreater{}}
\NormalTok{\textless{}feed xmlns="http://www.w3.org/2005/Atom"\textgreater{}}
\NormalTok{  \textless{}!{-}{-} Updated feed content {-}{-}\textgreater{}}
\NormalTok{\textless{}/feed\textgreater{}}
\end{Highlighting}
\end{Shaded}

\textbf{Advantages}: - ✅ Real-time updates (typically \textless{} 1
second latency) - ✅ Efficient bandwidth usage (push only when changed)
- ✅ Standardized W3C protocol - ✅ Decentralized (no vendor lock-in) -
✅ Built-in security via HMAC signatures

\textbf{Disadvantages}: - ❌ Requires public callback URL (challenging
for clients behind NAT/firewalls) - ❌ Additional infrastructure for
webhook endpoints - ❌ Hub availability dependency - ❌ Not suitable for
mobile apps without backend infrastructure

\paragraph{\texorpdfstring{\textbf{3. Atom Publishing Protocol
(AtomPub)}}{3. Atom Publishing Protocol (AtomPub)}}\label{atom-publishing-protocol-atompub}

Atom also defines a \textbf{publishing protocol} (RFC 5023) for creating
and editing feed content.

\textbf{Protocol}: AtomPub (HTTP-based RESTful API)

\textbf{Operations}:

\begin{itemize}
\tightlist
\item
  \textbf{GET}: Retrieve feed or entry
\item
  \textbf{POST}: Create new entry
\item
  \textbf{PUT}: Update existing entry
\item
  \textbf{DELETE}: Remove entry
\end{itemize}

\textbf{Example - Creating an Entry}:

\begin{Shaded}
\begin{Highlighting}[]
\NormalTok{POST /blog/entries HTTP/1.1}
\NormalTok{Host: example.com}
\NormalTok{Content{-}Type: application/atom+xml;type=entry}

\NormalTok{\textless{}?xml version="1.0"?\textgreater{}}
\NormalTok{\textless{}entry xmlns="http://www.w3.org/2005/Atom"\textgreater{}}
\NormalTok{  \textless{}title\textgreater{}New Blog Post\textless{}/title\textgreater{}}
\NormalTok{  \textless{}content type="xhtml"\textgreater{}}
\NormalTok{    \textless{}div xmlns="http://www.w3.org/1999/xhtml"\textgreater{}}
\NormalTok{      \textless{}p\textgreater{}This is the content.\textless{}/p\textgreater{}}
\NormalTok{    \textless{}/div\textgreater{}}
\NormalTok{  \textless{}/content\textgreater{}}
\NormalTok{  \textless{}author\textgreater{}}
\NormalTok{    \textless{}name\textgreater{}Jane Smith\textless{}/name\textgreater{}}
\NormalTok{  \textless{}/author\textgreater{}}
\NormalTok{\textless{}/entry\textgreater{}}
\end{Highlighting}
\end{Shaded}

\textbf{Note}: AtomPub is primarily a publishing mechanism, not a
notification system, but it complements Atom's ecosystem.

\begin{center}\rule{0.5\linewidth}{0.5pt}\end{center}

\subsubsection{📊 Atom Data Summary}\label{atom-data-summary}

\textbf{Data Richness}: \textbf{High} - Formally specified with strict
validation - Rich metadata constructs (Person, Link, Category) - Strong
internationalization support (IRI-based identifiers) - Clear content
type semantics

\textbf{Notification Model}: \textbf{Pull and Push (WebSub Integrated)}
- Default: HTTP polling (pull) - Recommended: WebSub for real-time push
notifications - Strong standardization for push mechanisms - Publishing
protocol available (AtomPub)

\begin{center}\rule{0.5\linewidth}{0.5pt}\end{center}

\subsection{⚖️ Comparative Analysis}\label{comparative-analysis}

\subsubsection{📊 Data Structure
Comparison}\label{data-structure-comparison}

\begin{longtable}[]{@{}
  >{\raggedright\arraybackslash}p{(\linewidth - 4\tabcolsep) * \real{0.3478}}
  >{\raggedright\arraybackslash}p{(\linewidth - 4\tabcolsep) * \real{0.3913}}
  >{\raggedright\arraybackslash}p{(\linewidth - 4\tabcolsep) * \real{0.2609}}@{}}
\toprule\noalign{}
\begin{minipage}[b]{\linewidth}\raggedright
Aspect
\end{minipage} & \begin{minipage}[b]{\linewidth}\raggedright
RSS 2.0
\end{minipage} & \begin{minipage}[b]{\linewidth}\raggedright
Atom
\end{minipage} \\
\midrule\noalign{}
\endhead
\bottomrule\noalign{}
\endlastfoot
\textbf{Standardization} & Informal specification (UserLand) & Formal
IETF standard (RFC 4287) \\
\textbf{Required Fields} & \texttt{\textless{}title\textgreater{}},
\texttt{\textless{}link\textgreater{}},
\texttt{\textless{}description\textgreater{}}
(channel)\texttt{\textless{}title\textgreater{}} OR
\texttt{\textless{}description\textgreater{}} (item) &
\texttt{\textless{}id\textgreater{}},
\texttt{\textless{}title\textgreater{}},
\texttt{\textless{}updated\textgreater{}} (feed \& entry)Plus
\texttt{\textless{}author\textgreater{}} or
\texttt{\textless{}link\textgreater{}} \\
\textbf{Unique Identifiers} & \texttt{\textless{}guid\textgreater{}}
(optional, can be permalink) & \texttt{\textless{}id\textgreater{}}
(required, must be permanent IRI) \\
\textbf{Timestamps} & \texttt{\textless{}pubDate\textgreater{}},
\texttt{\textless{}lastBuildDate\textgreater{}} (RFC 822) &
\texttt{\textless{}updated\textgreater{}},
\texttt{\textless{}published\textgreater{}} (RFC 3339) \\
\textbf{Author Metadata} & Simple text or email string & Structured
Person construct (\texttt{\textless{}name\textgreater{}},
\texttt{\textless{}uri\textgreater{}},
\texttt{\textless{}email\textgreater{}}) \\
\textbf{Content Representation} &
\texttt{\textless{}description\textgreater{}} (HTML or text) &
\texttt{\textless{}content\textgreater{}} (text, HTML, XHTML, external)
+ \texttt{\textless{}summary\textgreater{}} \\
\textbf{Media Attachments} & \texttt{\textless{}enclosure\textgreater{}}
element & \texttt{\textless{}link\ rel="enclosure"\textgreater{}}
element \\
\textbf{Categorization} & \texttt{\textless{}category\textgreater{}}
(simple text) & \texttt{\textless{}category\textgreater{}} (term,
scheme, label) \\
\textbf{Extensibility} & XML namespaces (iTunes, Dublin Core) & Limited
namespace usage (prefers inline constructs) \\
\textbf{Validation} & Loose, permissive parsing & Strict schema
validation required \\
\textbf{Date Format} & RFC 822
(\texttt{Fri,\ 10\ Oct\ 2025\ 12:00:00\ GMT}) & RFC 3339
(\texttt{2025-10-10T12:00:00Z}) \\
\textbf{Multiple Links} & Single \texttt{\textless{}link\textgreater{}}
per item & Multiple \texttt{\textless{}link\textgreater{}} with
\texttt{rel} attributes \\
\textbf{Self-Reference} & No standard mechanism & Required
\texttt{\textless{}link\ rel="self"\textgreater{}} \\
\textbf{Internationalization} & Limited (XML \texttt{lang} attribute) &
Strong (IRI-based, structured language support) \\
\end{longtable}

\subsubsection{🔔 Notification Mechanism
Comparison}\label{notification-mechanism-comparison}

\begin{longtable}[]{@{}
  >{\raggedright\arraybackslash}p{(\linewidth - 4\tabcolsep) * \real{0.3478}}
  >{\raggedright\arraybackslash}p{(\linewidth - 4\tabcolsep) * \real{0.3913}}
  >{\raggedright\arraybackslash}p{(\linewidth - 4\tabcolsep) * \real{0.2609}}@{}}
\toprule\noalign{}
\begin{minipage}[b]{\linewidth}\raggedright
Aspect
\end{minipage} & \begin{minipage}[b]{\linewidth}\raggedright
RSS 2.0
\end{minipage} & \begin{minipage}[b]{\linewidth}\raggedright
Atom
\end{minipage} \\
\midrule\noalign{}
\endhead
\bottomrule\noalign{}
\endlastfoot
\textbf{Default Model} & Pull (HTTP polling) & Pull (HTTP polling) \\
\textbf{Pull Protocol} & HTTP GET & HTTP GET \\
\textbf{Change Detection} &
\texttt{\textless{}lastBuildDate\textgreater{}},
\texttt{\textless{}pubDate\textgreater{}},
\texttt{\textless{}guid\textgreater{}} &
\texttt{\textless{}updated\textgreater{}},
\texttt{\textless{}id\textgreater{}} \\
\textbf{HTTP Caching} & \texttt{\textless{}ttl\textgreater{}} hint +
HTTP headers & HTTP cache-control headers \\
\textbf{Legacy Push} & \texttt{\textless{}cloud\textgreater{}} element
(RSSCloud) & Not applicable \\
\textbf{Modern Push} & WebSub (via Atom namespace) & WebSub (native
\texttt{\textless{}link\ rel="hub"\textgreater{}}) \\
\textbf{Push Standardization} & No standard push mechanism & W3C WebSub
standard \\
\textbf{Push Adoption} & Low (RSSCloud obsolete) & Moderate (WebSub
growing) \\
\textbf{Publishing Protocol} & No standard & AtomPub (RFC 5023) \\
\textbf{Real-time Capability} & Limited (via WebSub integration) &
Strong (WebSub native) \\
\end{longtable}

\subsubsection{🎯 Key Differences
Summary}\label{key-differences-summary}

\paragraph{\texorpdfstring{\textbf{1. Philosophy and
Design}}{1. Philosophy and Design}}\label{philosophy-and-design}

\begin{itemize}
\tightlist
\item
  \textbf{RSS 2.0}: Pragmatic simplicity and backward compatibility

  \begin{itemize}
  \tightlist
  \item
    Evolved organically from earlier RSS versions
  \item
    Prioritizes ease of implementation
  \item
    Tolerant of variations and extensions
  \end{itemize}
\item
  \textbf{Atom}: Formal standardization and clarity

  \begin{itemize}
  \tightlist
  \item
    Designed from scratch as IETF standard
  \item
    Prioritizes unambiguous specification
  \item
    Strict validation requirements
  \end{itemize}
\end{itemize}

\paragraph{\texorpdfstring{\textbf{2. Data
Richness}}{2. Data Richness}}\label{data-richness}

\begin{itemize}
\tightlist
\item
  \textbf{RSS 2.0}:

  \begin{itemize}
  \tightlist
  \item
    ✅ Extensible via namespaces (especially iTunes for podcasts)
  \item
    ✅ Sufficient for most syndication use cases
  \item
    ❌ Less structured metadata
  \item
    ❌ Ambiguous semantics for some elements
  \end{itemize}
\item
  \textbf{Atom}:

  \begin{itemize}
  \tightlist
  \item
    ✅ Rich, structured metadata constructs
  \item
    ✅ Clear semantics for all elements
  \item
    ✅ Strong internationalization (IRI-based)
  \item
    ❌ More verbose XML structure
  \end{itemize}
\end{itemize}

\paragraph{\texorpdfstring{\textbf{3. Notification
Ecosystem}}{3. Notification Ecosystem}}\label{notification-ecosystem}

\begin{itemize}
\tightlist
\item
  \textbf{RSS 2.0}:

  \begin{itemize}
  \tightlist
  \item
    ✅ Universal pull-based compatibility
  \item
    ✅ Simple polling implementation
  \item
    ❌ No standard push mechanism (RSSCloud obsolete)
  \item
    ⚠️ WebSub support via Atom namespace integration
  \end{itemize}
\item
  \textbf{Atom}:

  \begin{itemize}
  \tightlist
  \item
    ✅ Native WebSub integration
  \item
    ✅ Clear discovery via
    \texttt{\textless{}link\ rel="hub"\textgreater{}}
  \item
    ✅ Publishing protocol (AtomPub)
  \item
    ❌ WebSub still requires additional infrastructure
  \end{itemize}
\end{itemize}

\paragraph{\texorpdfstring{\textbf{4. Adoption and
Ecosystem}}{4. Adoption and Ecosystem}}\label{adoption-and-ecosystem}

\begin{itemize}
\tightlist
\item
  \textbf{RSS 2.0}:

  \begin{itemize}
  \tightlist
  \item
    ✅ Dominant in podcasting (99\%+ of podcast feeds)
  \item
    ✅ Wide client support
  \item
    ✅ Extensive tooling and libraries
  \item
    ✅ iTunes extension is de facto standard
  \end{itemize}
\item
  \textbf{Atom}:

  \begin{itemize}
  \tightlist
  \item
    ✅ Preferred by many blog platforms (WordPress, Blogger)
  \item
    ✅ Used by Google services (YouTube, Blogger)
  \item
    ✅ Strong in general RSS readers
  \item
    ❌ Limited podcast ecosystem adoption
  \end{itemize}
\end{itemize}

\paragraph{\texorpdfstring{\textbf{5. Validation and
Compliance}}{5. Validation and Compliance}}\label{validation-and-compliance}

\begin{itemize}
\tightlist
\item
  \textbf{RSS 2.0}:

  \begin{itemize}
  \tightlist
  \item
    ⚠️ Loose specification allows variations
  \item
    ⚠️ Many ``valid'' RSS feeds deviate from spec
  \item
    ✅ Parsers typically very tolerant
  \end{itemize}
\item
  \textbf{Atom}:

  \begin{itemize}
  \tightlist
  \item
    ✅ Strict XML schema validation
  \item
    ✅ Clear error messages for invalid feeds
  \item
    ❌ Less tolerance for non-compliant feeds
  \end{itemize}
\end{itemize}

\paragraph{\texorpdfstring{\textbf{6. Use Case
Recommendations}}{6. Use Case Recommendations}}\label{use-case-recommendations}

\begin{longtable}[]{@{}
  >{\raggedright\arraybackslash}p{(\linewidth - 4\tabcolsep) * \real{0.2703}}
  >{\raggedright\arraybackslash}p{(\linewidth - 4\tabcolsep) * \real{0.5135}}
  >{\raggedright\arraybackslash}p{(\linewidth - 4\tabcolsep) * \real{0.2162}}@{}}
\toprule\noalign{}
\begin{minipage}[b]{\linewidth}\raggedright
Use Case
\end{minipage} & \begin{minipage}[b]{\linewidth}\raggedright
Recommended Format
\end{minipage} & \begin{minipage}[b]{\linewidth}\raggedright
Reason
\end{minipage} \\
\midrule\noalign{}
\endhead
\bottomrule\noalign{}
\endlastfoot
\textbf{Podcasting} & RSS 2.0 & Universal client support, iTunes
extensions \\
\textbf{Blog Syndication} & Either (slight preference for Atom) & Both
widely supported \\
\textbf{Real-time Updates} & Atom with WebSub & Native push
integration \\
\textbf{Complex Metadata} & Atom & Richer data structures \\
\textbf{Simple Implementation} & RSS 2.0 & Less strict validation,
easier parsing \\
\textbf{Formal Compliance} & Atom & IETF standard, clear
specification \\
\end{longtable}

\begin{center}\rule{0.5\linewidth}{0.5pt}\end{center}

\subsubsection{📈 Visual Summary}\label{visual-summary}

\begin{verbatim}
┌─────────────────────────────────────────────────────────────┐
│                    RSS 2.0 vs Atom                          │
├─────────────────────────────────────────────────────────────┤
│                                                             │
│  RSS 2.0                            Atom                    │
│  ├─ Simple, pragmatic              ├─ Formal, standardized  │
│  ├─ Loose validation               ├─ Strict validation     │
│  ├─ Namespace extensions           ├─ Inline constructs     │
│  ├─ Podcast dominance              ├─ Blog platforms        │
│  ├─ Pull-based (default)           ├─ Pull + WebSub         │
│  └─ RFC 822 dates                  └─ RFC 3339 dates        │
│                                                             │
│  Notification Models:                                       │
│  ┌──────────────┐     ┌──────────────┐                    │
│  │ HTTP Polling │◄────┤ Both Support │                    │
│  └──────────────┘     └──────────────┘                    │
│                                                             │
│  ┌──────────────┐     ┌──────────────┐                    │
│  │   RSSCloud   │     │    WebSub    │◄──── Atom Native   │
│  │  (Obsolete)  │     │ (W3C Standard)│                    │
│  └──────────────┘     └──────────────┘                    │
│       ▲                      ▲                              │
│       │                      │                              │
│  RSS (rare)           Both (growing)                        │
│                                                             │
└─────────────────────────────────────────────────────────────┘
\end{verbatim}

\begin{center}\rule{0.5\linewidth}{0.5pt}\end{center}

\subsection{References}\label{references}

\subsubsection{Official Specifications}\label{official-specifications}

\begin{enumerate}
\def\labelenumi{\arabic{enumi}.}
\item
  \textbf{RSS 2.0 Specification} - Harvard Berkman Center\\
  \url{https://cyber.harvard.edu/rss/rss.html}\strut \\
  \emph{The canonical RSS 2.0 specification defining channel structure,
  item elements, and extension mechanisms. Essential reference for RSS
  feed generation and parsing.}
\item
  \textbf{Atom Syndication Format (RFC 4287)} - IETF\\
  \url{https://tools.ietf.org/html/rfc4287}\strut \\
  \emph{IETF standard for Atom feeds, providing formal XML schema,
  element definitions, and validation requirements. The authoritative
  source for Atom implementation.}
\item
  \textbf{Atom Publishing Protocol (RFC 5023)} - IETF\\
  \url{https://tools.ietf.org/html/rfc5023}\strut \\
  \emph{Defines the AtomPub protocol for creating, editing, and deleting
  Atom feed entries via HTTP. Complements Atom syndication with
  publishing capabilities.}
\item
  \textbf{WebSub Specification} - W3C Recommendation\\
  \url{https://www.w3.org/TR/websub/}\strut \\
  \emph{W3C standard for real-time content distribution using pub/sub
  architecture. The modern approach to push notifications for both RSS
  and Atom feeds.}
\end{enumerate}

\subsubsection{Technical Standards}\label{technical-standards}

\begin{enumerate}
\def\labelenumi{\arabic{enumi}.}
\setcounter{enumi}{4}
\item
  \textbf{RFC 822 - Standard for ARPA Internet Text Messages}\\
  \url{https://tools.ietf.org/html/rfc822}\strut \\
  \emph{Date format specification used by RSS 2.0 (\texttt{pubDate},
  \texttt{lastBuildDate}). Understanding RFC 822 dates is essential for
  proper RSS timestamp handling.}
\item
  \textbf{RFC 3339 - Date and Time on the Internet: Timestamps}\\
  \url{https://tools.ietf.org/html/rfc3339}\strut \\
  \emph{Date format specification used by Atom (\texttt{updated},
  \texttt{published}). Provides unambiguous timestamp representation for
  Atom feeds.}
\item
  \textbf{RFC 3986 - Uniform Resource Identifier (URI): Generic
  Syntax}\\
  \url{https://tools.ietf.org/html/rfc3986}\strut \\
  \emph{URI syntax standard referenced by both RSS and Atom. Critical
  for understanding feed URLs, links, and identifiers.}
\item
  \textbf{RFC 3987 - Internationalized Resource Identifiers (IRIs)}\\
  \url{https://tools.ietf.org/html/rfc3987}\strut \\
  \emph{IRI specification used extensively in Atom for internationalized
  identifiers. Extends URI syntax to support non-ASCII characters.}
\end{enumerate}

\subsubsection{Namespace Extensions}\label{namespace-extensions-1}

\begin{enumerate}
\def\labelenumi{\arabic{enumi}.}
\setcounter{enumi}{8}
\item
  \textbf{iTunes Podcast RSS Namespace} - Apple Developer\\
  \url{https://help.apple.com/itc/podcasts_connect/\#/itcb54353390}\strut \\
  \emph{Apple's podcast-specific RSS extensions defining
  \texttt{itunes:*} elements. Essential for podcast feed creation and
  distribution to Apple Podcasts and other directories.}
\item
  \textbf{Media RSS Specification} - Yahoo! Developer Network
  (Archive)\\
  \url{http://www.rssboard.org/media-rss}\strut \\
  \emph{RSS extension for multimedia content, defining \texttt{media:*}
  elements for images, videos, and audio with rich metadata.}
\item
  \textbf{Dublin Core Metadata Initiative}\\
  \url{https://www.dublincore.org/specifications/dublin-core/dcmi-terms/}\strut \\
  \emph{Metadata vocabulary often used as RSS namespace extension for
  additional descriptive elements like \texttt{dc:creator},
  \texttt{dc:rights}, etc.}
\end{enumerate}

\subsubsection{Validation Tools}\label{validation-tools}

\begin{enumerate}
\def\labelenumi{\arabic{enumi}.}
\setcounter{enumi}{11}
\item
  \textbf{W3C Feed Validation Service}\\
  \url{https://validator.w3.org/feed/}\strut \\
  \emph{Official validator for RSS and Atom feeds, providing syntax
  checking and compliance verification. Essential tool for testing feed
  implementations.}
\item
  \textbf{RSS Board Validator}\\
  \url{http://www.rssboard.org/rss-validator/}\strut \\
  \emph{RSS-specific validation service maintained by the RSS Advisory
  Board. Checks RSS 2.0 compliance and provides detailed error reports.}
\end{enumerate}

\subsubsection{Protocol Documentation}\label{protocol-documentation}

\begin{enumerate}
\def\labelenumi{\arabic{enumi}.}
\setcounter{enumi}{13}
\item
  \textbf{HTTP/1.1 Specification (RFC 7231)} - IETF\\
  \url{https://tools.ietf.org/html/rfc7231}\strut \\
  \emph{HTTP protocol specification covering request methods, status
  codes, and caching. Fundamental for understanding feed retrieval and
  conditional requests.}
\item
  \textbf{HTTP Caching (RFC 7234)} - IETF\\
  \url{https://tools.ietf.org/html/rfc7234}\strut \\
  \emph{HTTP caching mechanisms including \texttt{ETag},
  \texttt{Last-Modified}, \texttt{If-Modified-Since}, and cache-control
  headers. Critical for efficient feed polling.}
\end{enumerate}

\subsubsection{Industry Resources}\label{industry-resources}

\begin{enumerate}
\def\labelenumi{\arabic{enumi}.}
\setcounter{enumi}{15}
\item
  \textbf{RSS Advisory Board}\\
  \url{http://www.rssboard.org/}\strut \\
  \emph{Organization maintaining RSS specifications and best practices.
  Provides clarifications and guidance on RSS implementation.}
\item
  \textbf{Podcast Index Namespace} - Podcast Index\\
  \url{https://github.com/Podcastindex-org/podcast-namespace}\strut \\
  \emph{Modern podcast-specific RSS extensions including transcripts,
  chapters, value-for-value, and location data. Represents evolving
  podcast feed capabilities.}
\item
  \textbf{Feed Autodiscovery (RFC 5785)} - IETF\\
  \url{https://tools.ietf.org/html/rfc5785}\strut \\
  \emph{Defines well-known URIs for feed discovery, enabling clients to
  locate feeds from website URLs automatically.}
\end{enumerate}

\subsubsection{Historical Context}\label{historical-context}

\begin{enumerate}
\def\labelenumi{\arabic{enumi}.}
\setcounter{enumi}{18}
\item
  \textbf{``The Myth of RSS Compatibility''} - Mark Pilgrim (Archive)\\
  \url{https://web.archive.org/web/20110726121600/http://diveintomark.org/archives/2004/02/04/incompatible-rss}\strut \\
  \emph{Historical perspective on RSS evolution and compatibility issues
  that led to Atom's creation. Essential for understanding the
  philosophical differences.}
\item
  \textbf{``Why Atom 1.0?''} - Tim Bray (Archive)\\
  \url{https://www.tbray.org/ongoing/When/200x/2005/07/15/Atom-1.0}\strut \\
  \emph{Rationale for Atom's design decisions and improvements over RSS.
  Written by one of Atom's primary authors.}
\end{enumerate}

\subsubsection{Open Source
Implementations}\label{open-source-implementations}

\begin{enumerate}
\def\labelenumi{\arabic{enumi}.}
\setcounter{enumi}{20}
\item
  \textbf{Universal Feed Parser} - Python Library\\
  \url{https://github.com/kurtmckee/feedparser}\strut \\
  \emph{Popular Python library supporting RSS and Atom parsing.
  Excellent reference implementation demonstrating practical feed
  handling.}
\item
  \textbf{Rome} - Java RSS/Atom Library\\
  \url{https://github.com/rometools/rome}\strut \\
  \emph{Comprehensive Java library for RSS and Atom feed parsing and
  generation. Shows enterprise-grade feed processing.}
\item
  \textbf{Syndication (System.ServiceModel.Syndication)} - .NET\\
  \url{https://docs.microsoft.com/en-us/dotnet/api/system.servicemodel.syndication}\strut \\
  \emph{Microsoft's .NET framework classes for RSS and Atom feed
  handling. Official implementation for .NET applications.}
\end{enumerate}

\subsubsection{Research and Analysis}\label{research-and-analysis}

\begin{enumerate}
\def\labelenumi{\arabic{enumi}.}
\setcounter{enumi}{23}
\item
  \textbf{``RSS and Atom Compared''} - IBM developerWorks (Archive)\\
  \url{https://web.archive.org/web/20180808013923/https://www.ibm.com/developerworks/library/x-atom10/index.html}\strut \\
  \emph{Technical comparison of RSS and Atom from IBM's developer
  resources. Provides practical insights into choosing between formats.}
\item
  \textbf{``The Evolution of Web Syndication''} - ACM Queue\\
  \url{https://queue.acm.org/detail.cfm?id=1036497}\strut \\
  \emph{Academic perspective on syndication format evolution and the
  forces that shaped RSS and Atom development.}
\end{enumerate}

\subsubsection{Platform-Specific
Documentation}\label{platform-specific-documentation}

\begin{enumerate}
\def\labelenumi{\arabic{enumi}.}
\setcounter{enumi}{25}
\item
  \textbf{WordPress Feed Documentation}\\
  \url{https://wordpress.org/support/article/wordpress-feeds/}\strut \\
  \emph{Documentation for WordPress's RSS and Atom feed implementation,
  showing practical application in major CMS.}
\item
  \textbf{Google Reader API Documentation (Archive)}\\
  \url{https://web.archive.org/web/20130701000000*/https://developers.google.com/google-apps/reader/}\strut \\
  \emph{Historical documentation from Google Reader, demonstrating
  enterprise-scale feed aggregation architecture.}
\end{enumerate}

\begin{center}\rule{0.5\linewidth}{0.5pt}\end{center}

\emph{Document created: October 10, 2025 \textbar{} Version: 1.0}\\
\emph{Part of the Feed Architectures and Protocols series}




\end{document}
